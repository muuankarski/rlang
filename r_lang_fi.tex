\section{Avoimet tutkimusmenetelmät
yleisesti}\label{avoimet-tutkimusmenetelmuxe4t-yleisesti}

Uusi kappale @wiki\_avoin\_2013 sanoo tätä.

Keskustelu tutkimusmenetelmien avoimuudesta on ollut vähäistä
yhteiskuntatieteiden ja humanististen tieteiden (YHT) piirissä.
Keskeisin syy lienee se, että valtaosa YHT-tutkimuksesta on
\emph{laadullista}, jossa ei käytetä laskennallisia analyysimenetelmiä.
Ajatus siitä että \emph{diskurssianalyysin} käyttäminen velvoittaisi
tiukkojen lisenssiehtojen hyväksymistä ja rahallisen maksun
suorittamista on varmasti monelle vieras. Toisaalta määrällisessä
tutkimuksessa analyysiohjelmistojen valintaa ovat ohjanneet muut seikat
(perinteet, tuttuus) jolloin käytetyn ja opetetun analyysikielen
valinnan eri ulottuvuuksia ja pitkäaikaisia vaikutuksia ei olla
arvioitu. Joka tapauksessa 1) laskennallisen analyysin YHT-tieteellisten
sovellusmahdollisuuksien lisääntymisen sekä 2) avointen
analyysimenetelmien kehittymisen myötä on tutkimusmenetelmien avoimuus
on nousemassa yhä tärkeämmäksi puheenaiheeksi.

\subsection{Avoimen koodin
kehittyminen}\label{avoimen-koodin-kehittyminen}

Avoimilla tutkimusmenetelmillä tarkoitetaan laskennallisia
ohjelmointikieliä, joiden on lähdekoodi on avointa ja jotka on
lisensointu vapaasti. Vapaan ja avoimen lähdekoodin ohjelmistokehityksen
ideana on että kuka tahansa voi muokata lähdekoodia haluamallaan tavalla
sekä kopioida ja levittää sekä alkuperäistä että muokattua versiota ja
käyttää ohjelmaa mihin tahansa tarkoitukseen {[}@wiki\_avoin\_2013{]}.

Avointen tutkimusmenetelmien kehityksen arvioiminen on vaikeaa.
Menetelmien käytön yleisyys on yksi tapa, johon paneudutaan tekstin
toisessa luvussa r-kielen osalta. Avoimen koodin ympärille rakentuneet
projektit ovat tulleet yhä paremmin näkyville. Kenties tunnetuin
avoimeen lähdekoodin pohjautuva projekti on mobiililaitteiden
\emph{Android}-käyttöjärjestelmä, joka käyttää
\href{http://en.wikipedia.org/wiki/Android_\%28operating_system\%29\#Linux}{linux-ydintä},
on
\href{http://en.wikipedia.org/wiki/Android_\%28operating_system\%29\#Licensing}{avoimesti
lisensointu}, ja josta on lyhyessä ajassa tullut
\href{http://en.wikipedia.org/wiki/Android_\%28operating_system\%29\#Market_share_and_rate_of_adoption}{mobiilikäyttöjärjestelmien
markkinajohtaja}. Avoimeen lähdekoodiin perustuvat myös
internet-palvelimissa suositut linux-käyttöjärjestelmät, samoin kuin
niille talletettua tietoa hallinnoivat tietokantaohjelmat
(MySql,MariaDB). Nopeasti yleistyvät julkaisujärjestelmät, kuten
wordpress, drupal, tai joomla ovat myös suosittuja ja tavalliselle
käyttäjälle tuttuja avoimen lähdekoodin projekteja. Tieteellisessä
tutkimuksessa avoin lähdekoodin on ollut jo pitkään valtavirtaa. Yli
\href{http://en.wikipedia.org/wiki/Usage_share_of_operating_systems\#Supercomputers}{95
\% maailman supertietokoneista toimii linux-käyttöjärjestelmällä},
\emph{joku esimerkki geotieteistä} ja \emph{esimerkki lääketieteestä}.
Tilastollisen analyysin

Yhteiskuntatieteissä kuitenkin luotetaan edelleen siihen että
kaupalliset ohjelmistoratkaisut (esim. Stata,SPSS,SAS,Matlab, jne.) ovat
paras tapa vastata muuttuvan yhteiskunnan ja muuttuvan
yhteiskuntatieteen analyysitarpeisiin.

\subsection{Laskennallisen analyysin sovellusmahdollisuuksien
lisääntyminen}\label{laskennallisen-analyysin-sovellusmahdollisuuksien-lisuxe4uxe4ntyminen}

Laskennallisen analyysin soveltamismahdollisuudet YHT-tutkimuksessa ovat
lisääntyneet aineistojen digitalisoitumisen kautta. New York Times
{[}@lohr\_literary\_2013{]} käsittelin keväällä artikkelissaan
\emph{Dickens, Austen and Twain, Through a Digital Lens} Matthew L
Jockersin tuoretta kirjaa \emph{Macroanalysis Digital Methods and
Literary History} {[}@jockers\_macroanalysis\_2013{]}. Kirjassaan
Jockers analysoi vuosien 1780 - 1900 aikana 3592 englanniksi
kirjoitettua novellia data-analyysiä ja tilastollisia menetelmiä
yhdistelevällä metodilla paljastaen utta tietoa vaikutteiden
leviämisestä sekä yksittäisten kirjailijoiden merkityksestä
englantilaisen kaunokirjallisuuden historiassa.

@lohr\_literary\_2013 jatkaa artikkelissaan tulevaisuuden visiointiaan
olettaen että digitaaliset analyysimenetelmät tulevat leviämään
internet-teollisuudesta ja luonnontieteistä yhä laajemmalle
YHT-tutkimukseen.

\begin{quote}
Uudet löytämisen työkalut tarjoavat tuoreen näkökulman kulttuuriin;
pitkälti samaan tapaan kuin miten mikroskooppi auttoi näkemään elämän
hienorakenteet ja miten teleskoopit avasivat näkymät kaukaisiin
galakseihin.
\end{quote}

Harvardin yliopiston \href{http://www.iq.harvard.edu/}{\emph{Institute
for Quantitative Social Science}}:n johtaja professori
\href{http://gking.harvard.edu/}{Gary King} korostaa artikkelissaan
\emph{Restructuring the Social Sciences: Reflections from Harvard's
Institute for Quantitative Social Science} {[}s. 3,
@king\_restructuring\_2013{]} uuden avoimen datan merkitystä
perinteiselle määrälliselle yhteiskuntatutkimukselle:

\begin{quote}
An important driver of the change sweeping the field is the enormous
quantities of highly informative data inundating almost every area we
study. In the last half-century, the information base of social science
research has primarily come from three sources: survey research, end of
period government statistics, and one-off studies of particular people,
places, or events. In the next half-century, these sources will still be
used and improved, but the number and diversity of other sources of
information are increasing exponentially, and are already many orders of
magnitude more informative than ever before.
\end{quote}

Terveyden ja hyvinvoinnin laitoksen tutkimusprofessori Jussi
@simpura\_nakymattomien\_2012 kirjottaa Tiedepolitiikka lehdessä samasta
aiheesta:

\begin{quote}
Näyttää siltä, että tietotekniikan kehittymisen seurauksena
tietovarannot ovat räjähtämässä globaaliksi pilveksi ja että
ketjureaktio on jo käynnissä. Se koskettaa yksittäisiä tietovarantoja ja
pakottaa niitten kanssa työskenteleviä miettimään uusia toimintatapoja.
Avoimen datan aalto etenee globaalin tietopilven tuntumassa. Suomessakin
aalto on jo liikkeellä: vuonna 2012 avoimen datan edistämisen ja
hyödyntämisen ympärillä on ollut tapahtumia lähes viikottain ja
etenemisvauhti on kova. Avoimen datan pelisääntöjä rakenetaan
tilanteessa, jossa avoimen datan aalto uhkaa edetä sääntelijöiden
tavoittamattomiin. Tällä kaikella alkaa olla kiire, ja tiedemaailmankin
on pysyttävä aallon harjalla.
\end{quote}

\ldots{}Tilastollisen tutkimuksen käsitettä käytetään suomen kielessä
usein melko laveasti. Tässä tekstissä puhutaan laskennallisesta
analyysistä, joka nähdään koostuvan yhtäältä
tietojenkäsittelytieteellisestä data-analyysistä että toisaalta
tilastotieteellisestä data-analyysistä. Määrällisen tutkimusprosessin
työmäärästä usein vain pieni osa on varsinaista tilastotieteellistä
analyysiä ja suurin osa työstä kuluu datan keräämiseen sekä
muokkaamiseen.

Matthew L. Jonckers

\begin{center}\rule{3in}{0.4pt}\end{center}

@manovich\_software\_2013 {[}p. 22-23{]} sanoo että

\textbf{Tässä esseessä} käsittelen avointa sähköistä dataa
yhteiskuntatieteellisen tutkimuksen näkökulmasta. Avoin data tai
\emph{big data} tarkoittaa tässä yhteydessä verrattain uutta sähköiseen
muotoon kertyvää dataa, jota ei kerätä jotain tiettyä tutkimuskysymystä
silmällä pitäen, vaan data ikäänkuin \emph{kertyy} esimerkiksi
dokumentaation tai teknisen sovelluksen keräämänä. Tyypillisinä
esimerkkeinä aineistoista voidaan pitää mm. \emph{sosiaalista mediaa}
{[}esim. @king\_how\_2012{]}, \emph{vaalidataa} {[}esim.
@louhos\_2013{]}, \emph{poliittisen päätöksenteon dataa} {[}esim.
@hetherington\_how\_2012{]} tai vaikkapa \emph{rikollisuuteen ja
oikeuslaitoksen toimintaan liittyvää dataa} {[}esim.
@lipsky\_racial\_2012{]}. Kaikille näille tutkimuksille yhteistä on
avoiminen tietovirtojen tietotekninen hyödyntäminen tieteenalan
perinteisiin tutkimuskysymyksiin vastaamiseksi.

\textbf{Suomalainen määrällinen yhteiskuntatutkimus} on edelleen
vahvasti kiinni Kingin mainitsemissa \emph{menneen puolivuosisadan}
aineistoissa sekä menetelmissä. Menetelmällinen kehitys, sikäli kun sitä
on ollut, on painottunut uusien \emph{tilastollisten} menetelmien
soveltamiseen aineistojen ja tutkimuskysymysten pysyttyä pitkälti
ennallaan. Aineistojen osalta \textbf{viranomaisrekistereitä} on saatu
tutkittavaksi yhä enemmän. Rekisteriaineistotutkimus on
menetelmällisesti hyvin erityinen tutkimusala, jolla ei ole sovelluksia
juuri pohjoismaita kauemmaksi.

Uudet sähköiset aineistot eivät ole ainoa syy päivittää teoreettista ja
menetelmällistä osaamista. Etenkin survey-aineistojen käyttöä uhkaa
yhtäältä resurssien väheneminen ja toisaalta ihmisten vähenevä
osallistumisinto. Säästöpaineiden alla laajojen survey-aineistojen on
uhattuna, etenkin kun uusien kyselytutkimusten vastausprosentit ovat
tasolla, joka herättää kysymyksiä tulosten yleistämisestä ja siten koko
tutkimusmenetelmän käyttökelpoisuudesta tulevaisuudessa.

\textbf{Siirtyminen tulevan puolivuosisadan} aineistojen hyödyntämiseen
tulee olemaan kivinen tie. Se edellyttää merkittävää paradigma-tason
muutosta läpi tieteenalan. Käytännössä muutos edellyttää tietotekniikan
hyödyntämisen nostamista aivan uudelle tasolle. Menneen ajan aineistojen
analysointiin on voinut hankkia yksityiseltä yritykseltä sovelluksen
(esim. SPSS, Stata, SAS, jne.), jonka avulla kriteerit täyttävän mallin
on voinut suorittaa. Nykyään tieteelliselle tutkimukselle relevantit
uudet aineistot ja uudet analyysimenetelmät kehittyvät niin nopeasti,
etteivät kaupalliset tuotteet edes yritä pysyä kehityksessä mukana, vaan
keskittyvät palvelemaan yrityksiä joskin osin samoissa ison datan
haasteissa.

Tieteellisten ohjelmistojen kehitys onkin siirtynyt yhä enemmän
tiedeyhteisön varaan ja kehitystyöhön osallistumisesta tullut myös yksi
keskeinen akateemisen meritoitumisen muoto. Viime vuosina
yhdysvaltalaisten huippuyliopistojen professorirekrytoinneissa on
toistuvasti painotettu aktiivisuutta avoimen koodin
analyysiohjelmistojen kehitysyhteisöissä.

Meritoitumisen näkökulmasta uudet aineistot ja menetelmien kehittäminen
ovat kiinnostavia myös ns. tietovirtojen \emph{ohjaamisen} kannalta.
Tietovirroilla voidaan tarkoittaa esimerkiksi Venäjän
tilastoviranomaisen Rosstatin jatkuvasti julkaisevaa uutta tilastotietoa
tai vaikkapa Twitteriä. Mikäli tutkija onnistuu kehittämään nokkelan
tavan hyödyntää tätä tietovirtaa ja jakaa menetelmänsä avoimesti
tiedeyhteisön kesken, siirtyvät tutkijat helposti käyttämään tätä
menetelmää päästäkseen käsiksi ko. tietoihin. Tietovirtojen
kanavoinnista on tulossa myös yksi yliopistojen vetovoimaisuutta lisäävä
tekijä. Sama logiikka on \href{http://www.r-project.org/}{R-kielen}
suosion \href{http://r4stats.com/articles/popularity/}{nopean kasvun
takana}. R:n suosiosta kertoo paljon myös se, että parhaillaan
\href{https://www.coursera.org/}{Courserassa} käynnissä olevalle
\emph{Computing for Data Analysis}-kursille oli ilmoittautunut yli 40
000 opiskelijaa.

\textbf{Internetissä jaetun datan} lisääntyminen ja kiinnostus sen
hyödyntämiseen ovat yksi syy akateemisen julkaisutoiminnan ympärillä
tapahtuneeseen liikehdintään. Avoimen tiedon helppo saatavuus on
herättänyt kysymyksen siitä, miksi tutkijat eivät voisi yhtä lailla
jakaa omia tutkimustuloksiaan avoimesti ilman tai minimaalisin
välikäsin. \emph{Elsevier}-kustannustalon liiketoimintamallin
kritisoinnista lähtenyt
\href{http://en.wikipedia.org/wiki/Academic_Spring}{Academic
Spring}-liike on kuluneen vuoden aikana yhdistänyt laajasti eri maiden
tutkijayhteisöjä kyseenalaistamaan nykyistä akateemista julkaisumallia
(ks. \href{http://thecostofknowledge.com/}{\emph{Cost of knowledge}}).
Kritiikin kärki kohdistuu siihen, että isot kustantamot myyvät kovaan
hintaan takaisin yliopistoille tutkijoiden veronmaksajien rahoilla
tekemiä julkaisuja ja samalla pidättävät tekijänoikeudet itsellään
rajoittaen tutkijoiden mahdollisuuksia jakaa omia tuloksiaan vapaasti
netissä. Liikehdintä sai alkunsa matemaatikko Timothy Gowersin
\href{http://gowers.wordpress.com/2012/01/21/elsevier-my-part-in-its-downfall/}{blogissaan
julkaisemasta tekstistä}. Blogin kommenteissa eräs lukija nasevasti
tiivisti järjestelmän epäkohdat:

\begin{quote}
It is utterly absurd that we still have publishers --- we write for free
(because we want our work read or known), we edit or referee for free
and then pay large amounts of money to buy the work back. With the
advent of the Web, authors should have eliminated publishers.
\end{quote}

\textbf{Perinteinen akateeminen} julkaisumalli on monella tapaa jos ei
tiensä päässä. Toiminnan liiketaloudellinen kannattavuus on monella
tapaa uhattuna. Tutkimusjulkaisujen kysyntää siirtyy paperille
painetusta sähköisiin dokumentteihin, joiden tuotannossa ei enää tarvita
samanlaisia isoja pääomia kuin mitä esim. paperikirjojen. Uudet laitteet
(tietokoneet, tabletit, älypuhelimet) yleistyvät nopeasti ja ovat
osoittautuneet melko hyviksi artikkelikasojen korvaajiksi. Samalla myös
kiinnostus \href{http://en.wikipedia.org/wiki/Open-access_journal}{open
access -lehdet} lehtiä kohtaan on lisääntynyt, olkookin että niiden arvo
on vielä kaukana arvostettujen perinteisten lehtien tasosta. Kun taas
ajatellaan kirjoja niin kustannustaloilla on vielä myös niiden
\emph{nimi} tai \emph{maine}, jonka houkuttelemana ne säilyttävät
asemansa vielä pitkään. Siitäkin huolimatta, että teknologinen kehitys
on jo mitätöinyt kaikki muut kustantajien perinteiset vahvuudet.

Määrällisen tutkimuksen näkökulmasta sähköinen julkaiseminen on noussut
yhä kiinnostavammaksi ns. vuorovaikutteisen grafiikan kehittymisen
myötä. Grafiikan lisäksi on mahdollista luoda myös laskennallisia
analyysiympäristöjä raporttien lomaan, joita voi käyttää suoraan
selaimesta käsin. Varsinaisiin julkaisuihin vuorovaikutteisella
grafiikalla tai virtuaalisilla laskentaympäristöillä on vielä matkaa,
mutta erilaisissa \emph{harmaissa julkaisuissa} ja itse
tutkimusprosessissa ne tulevat olemaan hyödyksi jo lähitulevaisuudessa.

\textbf{Uusien teknologisten sovellusten} ohella sähköistä
julkaisumuotoa puoltavat myös etenkin laskennallisten tieteiden
kohtaamat vaatimukset analyysien \emph{toistettavuudesta} {[}ks. esim.
@peng\_reproducible\_2011{]}. Toistettavuudella tarkoitetaan datan ja
analyysialgoritmien avointa julkaisemista siten, että lukijan on
mahdollista arvioida analyysin pätevyyttä ja johtopäätöksiä.
Toistettavuudesta on alettu puhua enemmän myös yhteiskuntatieteellisen
tutkimuksen parissa kun samojen isojen kansainvälisten
tutkimusaineistojen parissa työskentelee tuhansia tutkijoita, joiden
tutkimustulokset ja johtopäätökset ovat usein ristiriitaisia.

\textbf{Palaan vielä johdannossa} viittaamaani New York Times
artikkeliin laskennallisista tutkimusmenetelmistä
kirjallisuudentutkimuksessa {[}@lohr\_literary\_2013{]}. Artikkelissa
pohdittiin paljon myös sitä, tarkoittaako \emph{määrän} lisääntyminen
laadullisen tutkimuksen aseman heikentymistä. Professori Jonckerin
mukaan oikeiden tutkimuskysymysten löytäminen ja analyysin virheiden
näkeminen edellyttävät syvää perehtyneisyyttä tieteenalaan yhä edelleen.

\begin{quote}
Quantitative tools in the humanities and the social sciences, as in
other fields, are most powerful when they are controlled by an
intelligent human. Experts with deep knowledge of a subject are needed
to ask the right questions and to recognize the shortcomings of
statistical models.
\end{quote}

\begin{quote}
``You'll always need both,'' says Mr.~Jockers, the literary quant. ``But
we're at a moment now when there is much greater acceptance of these
methods than in the past. There will come a time when this kind of
analysis is just part of the tool kit in the humanities, as in every
other discipline.''
\end{quote}

Yksittäisen tutkijan on vaikea vastata tähän haasteeseen ja määrällisen
analyysin yleistyminen tuleekin vaatimaan kaikilla tieteenaloilla
uudenlaista yhteistyötä tutkijoiden ja tieteenalojen kesken. Gary Kingin
{[}s 3., @king\_restructuring\_2013{]} mukaan myös Yhdysvalloissa
yhteiskuntatieteiden tutkijat ovat jo siirtymässä tutkijankammioista
yhteistyöhön kannustaviin, laboratorio-tyyppisiin monitieteisiin
tutkimusryhmiin. Laajan sähköisen datan analysointi edellyttää tietoa ja
osaamista, jota ei löydy miltään yhteiskuntatieteiden perinteiseltä
tieteenalalta.

\begin{quote}
Through collaboration across fields, however, we can begin to address
the interdisciplinary substantive knowledge needed, along with the
engineering, computational, ethical, and informatics challenges before
us. {[}s 3., @king\_restructuring\_2013{]}
\end{quote}

Uusi digitaalinen data myös hämärtää yhteiskuntatieteissä perinteistä
jakoa laadullisiin ja määrällisiin tutkimusotteisiin. Gary King {[}s 4.,
@king\_restructuring\_2013{]} ennustaa molempien tutkimussuuntien
yhdistyvän yhdeksi yhteiskuntatieteeksi, jossa samoja ongelmia ratkotaan
yhteistyössä.

\begin{quote}
Instead of quantitative researchers trying to build fully automated
methods and qualitative researchers trying to make due with traditional
human-only methods, both now are heading toward, using, or developing
computer-assisted methods that empower both groups. This development has
the potential to end the divide, to get us working together to solve
common problems, and to greatly strengthen the research output of social
science as a whole.
\end{quote}

Tällainen lähetyminen on ainakin suomalaisessa yhteiskuntatieteessä
vielä utopistinen visio. Sähköisen datan yhteiskunnallisen hyödyntämisen
kenties kiinnostavin avaus on Aalto-yliopiston ja Helsingin yliopiston
tietojenkäsittelytieteen jatko-opiskelijoiden perustama
\href{http://louhos.github.com/}{Louhos-projekti}. Hyvin kuvaavaa on se,
ettei tässä projektissa ole lainkaan yhteiskuntatieteilijöitä, ei
määrällisesti tai laadullisesti orientoituneita.

\textbf{Venäjän ja Itä-Euroopan} yhteiskuntatieteellisen tutkimuksen
näkökulmasta uudet avoimet aineistot ja menetelmät ovat erityisen
ajankohtaisia. Venäjällä viralliset tilastot ovat olleet avoimia vasta
muutaman vuoden ajan ja tekniset ratkaisut datan avaamiselle ovat
edelleen hyvin takapajuisia. Voidaankin sanoa että jo \emph{menneen
puolivuosisadan} aineistojen tehokas käyttö edellyttää Venäjän kohdalla
verrattain hyviä \emph{tulevan puolivuosisadan} menetelmien osaamista.

Samaan aikaan internetin merkitys kaikessa yhteiskunnallisessa näyttää
voimistuvan. Sosiaalisen median rooli yhteiskunnallisen keskustelun ja
jopa mobilisaation kanavana näyttää voimistuvan kaikkialla entisen
Neuvostoliiton alueella. Ja yhteiskunnan tutkijan tulee tietysti olla
paikalla siellä, missä yhteiskuntaa tehdään.

\section{Mikä on R? - Synty ja
ominaispiirteet}\label{mikuxe4-on-r---synty-ja-ominaispiirteet}

\begin{quote}
R has already won praise and plaudits from established media outlets
such as the New York Times, Forbes, Intelligent Enterprise, InfoWorld
and The Register. When you consider that R is a high-level computer
programming language designed mostly for quants (the nickname for a
subspecies of geeks who focus on quantitative analysis), the adoring
media attention seems nothing short of astounding.
\end{quote}

Joka tapauksessa @smith\_r\_2010 {[}s.23{]} kirjoittaa että paska on
aina paskaa.

\subsection{Synty}\label{synty}

\begin{itemize}
\itemsep1pt\parskip0pt\parsep0pt
\item
  R-language was iniated by two
\item
  open source
\end{itemize}

\subsection{R-kieli
ohjelmointikielenä}\label{r-kieli-ohjelmointikielenuxe4}

\begin{itemize}
\itemsep1pt\parskip0pt\parsep0pt
\item
  object oriented, s-language bell laboratories (G)UI's, IDE Rstudio,
\item
  contributed packages
\end{itemize}

\subsection{R-kieli tilastollisen laskennan
työkaluna}\label{r-kieli-tilastollisen-laskennan-tyuxf6kaluna}

\begin{itemize}
\itemsep1pt\parskip0pt\parsep0pt
\item
  structure
\item
  visual
\item
  open source, community, licensing, teaching
\end{itemize}

\subsection{R-kielen suosio}\label{r-kielen-suosio}

\begin{itemize}
\itemsep1pt\parskip0pt\parsep0pt
\item
  enterprise level services
\end{itemize}

\section{Miten R on mahdollinen? -
R-projekti}\label{miten-r-on-mahdollinen---r-projekti}

\subsection{Projektin organisaatio}\label{projektin-organisaatio}

\begin{itemize}
\itemsep1pt\parskip0pt\parsep0pt
\item
  development vs.~user help
\end{itemize}

\subsection{Kielen kehitys}\label{kielen-kehitys}

\subsection{Käyttäjätuki}\label{kuxe4yttuxe4juxe4tuki}

\begin{itemize}
\itemsep1pt\parskip0pt\parsep0pt
\item
  mailing lists - general vs.~special interest groups blogs
\item
  q \& a sites
\end{itemize}

\subsection{Kontribuoidut paketit}\label{kontribuoidut-paketit}

\begin{itemize}
\itemsep1pt\parskip0pt\parsep0pt
\item
  CRAN - task views R-forge
\item
  Github
\item
  bioconductor
\end{itemize}

\section{Mihin R-kieltä käytetään? - R käytännön
työssä}\label{mihin-r-kieltuxe4-kuxe4ytetuxe4uxe4n---r-kuxe4ytuxe4nnuxf6n-tyuxf6ssuxe4}

Why R-language is popular

\subsection{Tilastollisten menetelmien
kehitys}\label{tilastollisten-menetelmien-kehitys}

\begin{itemize}
\itemsep1pt\parskip0pt\parsep0pt
\item
  new methods implemented first in R
\end{itemize}

\subsection{Soveltava tilastotiede}\label{soveltava-tilastotiede}

\subsubsection{Bio/geo-tieteet}\label{biogeo-tieteet}

\begin{itemize}
\itemsep1pt\parskip0pt\parsep0pt
\item
  Geograhical Information Systems
\end{itemize}

\subsubsection{Yhteiskuntatieteet}\label{yhteiskuntatieteet}

\subsubsection{Humanistiset tieteet - ihmiskielten
analyysi}\label{humanistiset-tieteet---ihmiskielten-analyysi}

\subsection{Yritysanalytiikka}\label{yritysanalytiikka}

\begin{itemize}
\itemsep1pt\parskip0pt\parsep0pt
\item
  insurance, big data, banking, industry
\item
  social media: facebook, google, twitter
\end{itemize}

\subsection{Datajournalismi}\label{datajournalismi}

\begin{itemize}
\itemsep1pt\parskip0pt\parsep0pt
\item
  Guardian, New York Times, Chicago Herald Tribune
\end{itemize}

\section{Miten R toimii käytännössä? - Yhteiskuntatieteellisten
aineistojen analyysi
R-kielellä.}\label{miten-r-toimii-kuxe4ytuxe4nnuxf6ssuxe4---yhteiskuntatieteellisten-aineistojen-analyysi-r-kielelluxe4.}

\subsection{Sanapilvet}\label{sanapilvet}

\subsection{Verkostokartat}\label{verkostokartat}

\subsection{Spatiaalinen
visualisointi}\label{spatiaalinen-visualisointi}

\subsection{Klusterointi}\label{klusterointi}

\begin{itemize}
\itemsep1pt\parskip0pt\parsep0pt
\item
  insurance, big data, banking, industry
\item
  social media: facebook, google, twitter
\end{itemize}

\section{Kirjallisuus}\label{kirjallisuus}
